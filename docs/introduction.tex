\section{Introduction}

TODO...

\plant{At} is a nonselective model organism that is known to accommodate different climates easily. In this study we want to find out if samples of \plant{At} in Spain, that have been under drought conditions, show different genetic diversity than of the samples in Sweden.

It is important to research plants’ drought resistance since it is a pressuring issue when it comes to agriculture. Water limitation is an ongoing challenge in agriculture and due to climate change it is significant to see how the plants adjust to the harsh climates and to investigate the effects of changing climate on these plants.

The findings of this study can shed light to which mechanisms and genes help plants against harsh conditions and also later help us in the research of other model organisms and plants.The findings can also contribute to understanding and improving plants against harsh conditions in agriculture via genetic engineering. We wanted to see if the drought would affect the genetic diversity of the plant.

Tahir:

I have reviewed your research project idea and find it to be a very relevant topic for exploration. This investigation can help you to understand how \plant{At} adapts to drought stress and the resulting impact on its genetic diversity. While comparing the genetic diversity of \plant{At} samples from Spain and Sweden, where plants face varying drought conditions, is a good starting point, consider refining your research question. Focus on drought stress-related genes that may show variation and adaptation in \plant{At} populations across different regions. Utilize measures like Tajima’s D or Fst to assess genetic diversity and adaptation, observing variations between populations from the two regions. Alternatively, you may choose to conduct a Genome-Wide Association Study (GWAS) to pinpoint SNPs and genes associated with drought tolerance or resistance.